\documentclass[a4paper]{article}

\usepackage[hangul]{kotex}

\author{고려대학교 세계와 한국경제 2018년 2학기 수강생들}
\title{Acemoglu "경제학원론" 2016년판 번역서(시그마프레스) 오류리스트}

\begin{document}

\maketitle
	
\paragraph{p.13: 12} % (fold)
\label{par:p_13_12}
다르게 우리는 균형을 누구도... 에서 "다르게"는 지워야 함 (2018-2 윤선우)
% section p_13_12 (end)

\paragraph{p.52:3-4} % (fold)
\label{par:p_52_3_4}
최소가치는 minimum value의 오역. "최소값"으로 번역해야 함
% section p_52_3_4 (end)

\paragraph{p.60: 6번문제 두번째 표} % (fold)
\label{par:p_60_6}
남쪽 주차장의 수치 정보가 누락되어 있음 (2018-2 곽동원)
% section p_60_6\xEB\xB2\x88\xEB\xAC\xB8\xEC\xA0\x9C_\xEB\x91\x90\xEB\xB2\x88\xEC\xA7\xB8_\xED\x91\x9C (end)

\paragraph{p.61: 8번문제 세번째 줄} % (fold)
\label{par:p_61_8}
\$30,000를 초과하면, 당신은 \$30,000을 초과하는 액수의 30\%를 세금으로.. (2018-2 곽동원)
% section p_61_8\xEB\xB2\x88\xEB\xAC\xB8\xEC\xA0\x9C_\xEC\x84\xB8\xEB\xB2\x88\xEC\xA7\xB8_\xEC\xA4\x84 (end)

\paragraph{p.76:왼쪽단} % (fold)
\label{par:p_76_}
투입물은 다른 재화나 서비스를 생산하기 위해 사용되는 재화나 서비스이다
(2018-2 백송하)
% section p_76_\xEC\x99\xBC\xEC\xAA\xBD\xEB\x8B\xA8 (end)

\paragraph{p.105: 호탄력성 박스 안 10번째줄} % (fold)
\label{par:p_105_}
\begin{itemize}
	\item 가격이 \$1 하락할 때가 아니라 가격이 \$1 로 하락할때임 (2018-2 윤선우)

	\item 호탄력성의 첫번째 수식 분자에 괄호가 빠져 있음 (2018-2 주하은)
\end{itemize}
% section p_105_\xED\x98\xB8\xED\x83\x84\xEB\xA0\xA5\xEC\x84\xB1_\xEB\xB0\x95\xEC\x8A\xA4_\xEC\x95\x88_10\xEB\xB2\x88\xEC\xA7\xB8\xEC\xA4\x84 (end)

\paragraph{p.113: 12번문제 1번째줄} % (fold)
\label{par:p_113_12}
감자에 대한 수요의 소득탄력성은 -0.5임. 원문에는 -0.5로 기술되고 있고 통상 감자는 열등재의 예로 많이 쓰임 (2018-2 권혁준)
% section p_113_12\xEB\xB2\x88\xEB\xAC\xB8\xEC\xA0\x9C_1\xEB\xB2\x88\xEC\xA7\xB8\xEC\xA4\x84 (end)

\paragraph{p.147} % (fold)
\label{par:p_147}
교과서 147p 12번 문제에서 b.번에서 '극소화된 지점에서 가장 낮은 산출량을 말한다'를 '극소화된 지점에서의 산출량을 말한다'로 수정하는 것이 좋을 것 같습니다. (2018-2 곽동원)
% paragraph p_147 (end)

\paragraph{p.176: 9번문제 6번째줄} % (fold)
\label{par:p_176_9}
마다가스카르로 옮기는 것이 아니라 일본으로 3,000쌍 생산을 옮기는 것을 검토하는 상황임 (2018-2 권혁준)
% section p_176_9\xEB\xB2\x88\xEB\xAC\xB8\xEC\xA0\x9C_6\xEB\xB2\x88\xEC\xA7\xB8\xEC\xA4\x84 (end)

\paragraph{p.199: 도표8.15} % (fold)
\label{par:p_199_}
캡션 마지막의 "사장손실"은 "사중손실"로 정정해야 함 (2018-2 이상구))
% section p_199_\xEB\x8F\x84\xED\x91\x9C8_15 (end)

\paragraph{p.204: 문제2의 표} % (fold)
\label{par:p_204_}
재닛의 수학 기회비용은 1 경제학문제가 아니라 1/2 경제학 문제임
(2018-2 곽동원)
% section p_204_\xEB\xAC\xB8\xEC\xA0\x9C2\xEC\x9D\x98_\xED\x91\x9C (end)

\paragraph{p.204: 문제1 2번째줄} % (fold)
\label{par:p_204}
"컵이크" $\Rightarrow$ "컵케이크" (2018-2 권혁준)
% section p_204 (end)

\paragraph{p.222: 3-4} % (fold)
\label{par:p_222_3_4}
교육을 더 받을지 여부를 결정할 때 당신은 교정적 보조금을 감안하기 때문에 당신은 외부효과를 고려한다 (당신은이 이중으로 들어가 있으므로 하나를 삭제해야 함) (2018-2 윤선우)
% section p_222_3_4 (end)

\paragraph{p.261} % (fold)
\label{par:p_261}
4번의 c항: "사장손실"이 아니라 "사중손실"임 (2018-2 박상우)
% paragraph p_261 (end)
\paragraph{p.265} % (fold)
\label{par:p_265}
책p265 마지막 단락, 밑에서 3번째 줄에 왜냐하면 인간은 기계에는 여전히 부족한 판단 기술을 가지고 있기 때문이다. 
이 부분이 번역 오류로 보이는데요
 문맥상 
1.왜냐하면 인간보다 기계는 여전히 부족한 판단 기술을 가지고 있기 때문이다.  
혹은 
2. 왜냐하면 인간은 기계보다 뛰어난 판단 능력을 가지고 있기 때문이다.
이 두가지가 적절해 보입니다.  (2018-2 김성준)
% paragraph p_265 (end)

\paragraph{p.310} % (fold)
\label{par:p_310}
책310쪽에 5번문제에서 "평균비용이 일정하고 한계비용이 6달러인" 이 부분을 아래 6번문제를 고려했을 때 평균비용과 한계비용이 6달러로 일정하다고 고쳐야 될 것 같습니다. (2018-2 김성준)
% paragraph p_310 (end)

\paragraph{p.310} % (fold)
\label{par:p_310}
책310쪽에 6번문제에서 "독점기업은 수요곡선과 절편은 같고" 이 부분을 독점기업은 수요곡선과 y절편은 같고 로 고쳐야 할것 같습니다.
전체 절편이 같다고 해버리면 기울기가2배가 될 수 없기 때문입니다. (2018-2 김성준)
% paragraph p_310 (end)

\paragraph{p.315} % (fold)
\label{par:p_315}
원론책 315p 우월전략과 우월전략 균형 파트 첫밴째와 두번째 단락에 '고백' 전략을 택하는 것으로 나오는데 제시된 전략은 '자백'과 '버티기'입니다. (2018-2 박영신)
% paragraph p_315 (end)
\paragraph{p.347} % (fold)
\label{par:p_347}
14장 과점과 독점적 경쟁관련해서 주교재 공부를 하다가 번역오류로 보이는 부분을 발견해 이메일 보내게 되었습니다. 

교재 p347의 마지막 단락 아래에서 5번째줄 "우리는 한 상품에 대한 대체재가 많아질수록 기업의 잔여수요함수는 왼쪽으로 이동하여 더 탄력적으로(덜 가파르게) 됨을 알고있다. "가 그 부분인데요,

우리는 한 상품에 대한 대체재가 많아질수록 기업의 잔여수요함수는 왼쪽으로 이동하고 또한 더 탄력적(덜 가파르게) 됨을 알고있다.라고 고쳐야 정확한 문장일 것 같습니다.

왜냐하면 한 상품에 대한 대체재가 많아지면 기업의 잔여수요 함수가 왼쪽으로 이동하고. 한 상품에 대한 대체재가 많아질수록 수요함수가 더 탄력적으로 되기 때문입니다.

교재의 본문장은 마치 논리학의 삼단논법처럼 한 상품에 대한 대체재가 많아지면 기업의 잔여수요함수는 왼쪽으로 이동하고, 즉 수요함수가 왼쪽으로 이동했기(수요가 감소했기 ) 때문에 더 탄력적으로 된다고 기술하여("잔여수요함수는 왼쪽으로 이동하여 더탄력적으로~")  오독의 가능성을 내포하고 있다고 생각됩니다. (2018-2 김성준)
% paragraph p_347 (end)
\paragraph{p.355} % (fold)
\label{par:p_355}
: 질문 4번 中 “단기에서 경제적 이윤을 얻는 독점적 경제기업이 장기에서도 이윤을 얻겠는가?”
독점적 경제기업이라는 워딩을 독점적 경쟁기업으로 수정하는 것이 옳다고 보입니다. (2018-2 임동언)
% paragraph p_355 (end)
\paragraph{p.448} % (fold)
\label{par:p_448}
: 위에서 네 번째 줄 中 “… 가계는 포드에게 무스탕 자동차에 대한 대가로 \$30,000를 지불한다. …”
바로 앞 장(p.447)에서 ‘머스탱’이라고 표기하고 있는 만큼 수정이 필요해보입니다. (2018-2 임동언)
% paragraph p_448 (end)
\end{document}